\documentclass[UTF8]{article}
\usepackage{fontspec} 
\usepackage{xltxtra} 
\usepackage{xeCJK}
\usepackage{fancyhdr}
\usepackage{xunicode}
\usepackage{ctex} 
\usepackage{xcolor}
\usepackage{listings}
\usepackage{geometry}
\usepackage{latexsym}
\author{miniLCT{}}
\pagestyle{fancy}
\rhead{ACM International Collegiate Programming Contest}
%\setCJKmainfont[BoldFont=SimHei]{SimSun}
%\setmainfont{qingsong} %主要字体
\title{算法竞赛进阶指南 代码参考} 

\newfontfamily\codefont{Source Code Pro}
\geometry{a4paper,scale=0.8,left=3.0cm,top=2.5cm,bottom=2.5cm} %格式
\lstset{ 
    rulesepcolor= \color{gray}, %代码块边框颜色
    tabsize=4,
    breaklines=true,  %代码过长则换行
    numbers=left, %行号在左侧显示
    numberstyle= \small,%行号字体
    showstringspaces=false,
    commentstyle=\color{gray}, %注释颜色
    frame=lines,
    basicstyle=\codefont,
    language=c++,
}
\begin{document}
\maketitle{}
\begin{figure}[h]
\centering
\includegraphics[scale=0.3]{a.png}
%\caption{The Universe}
%\label{fig:universe}
\end{figure}
\newpage
\tableofcontents
\newpage

%\part{Start}
\section{0x00 基本算法}
\subsection{0x01 位运算}
\subsubsection{快速幂}
\subparagraph{bin }
\begin{lstlisting}
LL bin(LL x, LL n, LL MOD) {LL ret = MOD != 1;for (x %= MOD; n; n >>= 1, x = x * x % MOD)if (n & 1) ret = ret * x % MOD;return ret;}
inline LL get_inv(LL x, LL p) { return bin(x, p - 2, p); }
\end{lstlisting}
\subsubsection{GCC内置函数}
\subparagraph{与机器或编译器版本有关,部分竞赛可能会禁止}

\begin{lstlisting}
int __builtin_ctz(unsigned int x)
int __builtin_ctzll(unsigned long long x)
返回x的二进制表示下最低位的1后面几个0
int __builtin_popcount(unsigned int x)
int __builtin_popcountll(unsigned long long x)
返回x的二进制表示下有多少位1
\end{lstlisting}


\section{0x10 基本数据结构}

\section{0x20 搜索}

\section{0x30 数学知识}

\section{0x40 数据结构结构}

\section{0x50 动态规划}

\section{0x60 图论}

\section{0x70 综合技巧与实践}


\end{document}
% Local Variables:
% TeX-engine: xelatex
% End: